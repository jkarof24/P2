
Regression is a tool in statistics used to understand the relationship between one dependent variable and one or more independent variables. In most research problems where regression analysis is applied, more than one independent variable is needed in the regression model. For the regression model to be both accurate and reliable, a set of assumptions needs to be met. But, it is not certain that real-life data supports these assumptions. Therefore, it is interesting to explore how to manage these assumption violations. \newline

\noindent There are different methods to handle assumption violations in regression models. This project explores the  Monte Carlo Bootstrap method, which is based on resampling. The focus of the project is based on the assumption of homoscedasticity and how to handle this assumption violation using bootstrapping.\newline

\noindent The structure of the project starts off by analysing the general problem when assumptions in regression models are violated. In Section 3 and 4, the necessary mathematical theory, as well as the chosen method and metrics, are explained. Next, the problem statement is presented. Finally, in Sections 5, 6, and 7, the problem is solved, and the project is discussed and concluded.