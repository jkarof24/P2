In data science and other fields of computer science, making predictions and understanding the relationships between variables is essential. Regression models are widely used tools for this purpose. However, to produce high-quality regression models, certain assumptions about the underlying data must be met. The challenge is that real-world data often violate these assumptions, which can lead to inaccurate and unreliable results. This makes it important to explore how such assumption violations can be managed effectively. \newline

\noindent One important assumption is homoscedasticity, which is the idea that the variance of the errors remain constant across the independent variables. If this assumption is violated, heteroscedasticity occurs, which leads to biased standard errors. This makes it difficult to draw valid conclusions from the model. \newline


\noindent There are different methods to handle assumption violations in regression models. This project investigates how Monte Carlo Bootstrapping can reduce the effects of assumption violations in polynomial regression, focusing on homoscedasticity. Synthetic data designed to violate this assumption will be used to compare the performance of models built using traditional methods versus bootstrapped models.\newline

\noindent The structure of the project starts off by analysing the general problem when assumptions in regression models are violated. In Section 4 and 5, the necessary mathematical theory, as well as the chosen method and metrics, are explained. Next, the problem statement is presented. Finally, in Sections 7, 8, and 9, the problem is solved, and the project is discussed and concluded.
