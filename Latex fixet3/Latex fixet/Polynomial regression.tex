\subsection{Regression Model}
Understanding the relationship between variables is crucial when working with data, as it forms the basis for drawing meaningful conclusions. Regression models are a fundamental statistical tool used to model and analyse these relationships. In this project the chosen model, we are working with is polynomial regression. This section will explain the neccessary theory to understand this model. It will also explain assumption violations in more detail. Specifically the assumption of constant variance of errors also called homoscedasticity and the assumption of no multicollinearity.

\subsubsection{Regression}
Sections $3.7.1$, $3.7.2$ and $3.7.3$ are based on the book 'Applied Lienar Statistical Models' \cite{AppliedLSM}. 
\newline 

\noindent Before looking at polynomial regression, we have to understand linear regression. \newline 
Linear regression is a model that estimates the relationship between a dependent variable, \( y \), and one or more independent variables, \( x \). A reasonable relationship between the two in simple regression is the linear relationship:
\begin{equation}
Y = \beta_0 + \beta_1 x .
\end{equation}


\noindent Where \( \beta_0 \) is the intercept, and \( \beta_1 \) is the slope.

\noindent In a lot of cases, there will be more independent variables, so the relationship for multiple regression will look like this:

\begin{equation}
	Y = \beta_0 + \beta_1 x_1 + ......+ \beta_n x_n .
\end{equation}




\noindent Where \( n \) is the number of independent variables, and $\beta_2$ further shapes the curvature and complexity of the curve. Linear models use the method of least squares of the residuals to estimate parameters, in order to find the best fitting line for the data.

\noindent In simple linear regression, the random error \( \epsilon \) is included:
\begin{equation}
Y = \beta_0 + \beta_1 x + \epsilon .
\end{equation}


\noindent It is assumed that \( \epsilon \) is distributed with $\epsilon = 0$ and $\epsilon) = \sigma^2$, and it has consistent variance, which is usually called the \textit{homogeneous variance assumption}. The random error \( \epsilon \) adds randomness to account for the natural variability in real data, making the model more realistic.
\newline\\
Polynomial regression is a form of linear regression, but the relationship between \( x \) and \( y \) is an \( n \)th-degree polynomial. It fits a nonlinear relationship between the value of \( x \) and the corresponding conditional mean of \( y \), meaning the model predicts the expected value of \( y \) given \( x \). \newline

\noindent That is why it is used when the relationship between the independent variable and the dependent variable is better represented by a curve rather than a straight line, since it can show the nonlinear patterns in the data.
In polynomial regression, as mentioned before, there are six assumptions, that should be met, for the model to be as accurate as possible.


\subsubsection{The Method of Least Squares}

\noindent To find connections in data, it is necessary to estimate coefficients, $\beta_0$ and $\beta_1$, in linear models. 
A widely used method to estimate the coefficients, is the previously mentioned least squares method, which will be referred to as Ordinary Least Squares or OLS in the rest of the project. \newline

\noindent OLS considers the deviation of $Y_i$ for  its expected value, where the observations are $(X_i, Y_i)$. 
This method also requires, that we consider the sum of the $n$ squared deviations.
This is denoted by the criterion $Q$,
\begin{equation}
Q=\sum_{i=1}^{n}(Y_1-\beta_0 - \beta_1 X_i)^2 .
\end{equation}

\noindent According to the method of OLS, the estimations of $\beta_0$ and $\beta_1$ that minimize Q for the given sample observations $(X_1,Y_1), (X_2,Y_2), ..., (X_n,Y_n)$, are called $b_0$ and $b_1$.  

\noindent If an analytical approach is used, the values $b_0$ and $b_1$ that minimize Q for any particular set of sample data are given by these simultaneous equations: 
\begin{equation}
\sum Y_i =n b_0 +b_1 \sum X_i ,
\end{equation}

\begin{equation}
\sum X_i Y_i = b_0 \sum X_i + b_1 \sum X_1^2 .
\end{equation}

\noindent These equations are called $normal equations$, where $b_0$ and $b_1$ are the $point estimators$ of $\beta_0$ and $\beta_1$. It is possible to calculate these normal equations simultaneously for $b_0$ and $b_1$ through these expressions,

\begin{equation}
b_1 = \frac{\sum (X_1 - \bar{X}) (Y_i - \bar{Y})}{\sum (X_i - \bar{X})^2} ,
\end{equation}

\begin{equation}
	b_0 = \frac{1}{n} (\sum Y_i - b_1 \sum X_i ) = \bar{Y} - b_1 \bar{X} .
\end{equation}

\noindent Here, $\bar{X}$ and $\bar{Y}$ are the means of $X_i$ and $Y_i$. The normal equations can also be derived by differentiating with respect to $\beta_0$ and $\beta_1$:

\begin{equation}
	\frac{\partial Q}{\partial \beta_0}=-2 \sum (Y_i - \beta_0 - \beta_1 X_i) ,
\end{equation}
\begin{equation}
\frac{\partial Q} {\partial \beta_1} = -2 \sum X_i (Y_i - \beta_0 - \beta_1 X_i) .
\end{equation}

\noindent By setting these derivatives equal to zero, we can find $b_0$ and $b_1$, 

\begin{equation}
	-2 \sum (Y_i - b_0 - b_1 X_1)=0 ,
\end{equation}
\begin{equation}
	-2\sum X_i(Y_i - b_0 - b_1 X_1)=0 .
\end{equation}

\noindent This can be simplified, 
\begin{equation}
\sum_{i=1}^{n} (Y_i - b_0 - b_1 Xi)=0£ ,
\end{equation}
\begin{equation}
	\sum_{i=1}^{n} X_1(Y_i - b_0 - b_1 Xi)=0£ .
\end{equation}
And it can be expanded, so the normal equations are obtained, 

\begin{equation}
	\sum Y_1 - n b_0 - b_1 \sum X_1 =0 ,
\end{equation}

\begin{equation}
	\sum X_1 Y_1  - b_0 \sum X_1 - b_1 \sum X_i^2 =0 .
\end{equation}

\noindent Solving this for $b_0$ and $b_1$ will lead to values, that minimize Q, and these are the estimates for $\beta_0$ and $\beta_1$. 
When rearranging terms, we get the normal equations $(28)$ and $(29)$. The estimates $b_0$ and $b_1$ obtain the minimum when checking the second partial derivatives. \newline


\subsubsection{Linear Models}
Equations for linear models can be written in matrix terms, where the normal error regression model for simple linear regression is.
 

\begin{equation}
Y_i = \beta_0 + \beta_1 X_i 0 \epsilon_i .
\end{equation}


\noindent Which implies that,

\begin{equation}
Y_1 = \beta_0 + \beta_1 X_1 + \epsilon_1
\end{equation}

\begin{equation}
Y_2 = \beta_0 + \beta_2 X_2 + \epsilon_2
\end{equation}
$$\vdots$$
\begin{equation}
Y_n = \beta_0 + \beta_n X_n + \epsilon_n .
\end{equation}

\noindent The observations vector $Y$ is,
\begin{equation} Y_{n \times 1} =
\left[
\begin{array}{c}
	Y_1 \\ 
	Y_2 \\ 
	\vdots \\
	Y_n 
\end{array}
\right].
\end{equation}

\noindent The X matrix is, 

\begin{equation}X_{n \times 2}=
\left[
\begin{array}{cc}
	1 & X_1 \\ 
	1 & X_2 \\ 
	\vdots & \vdots \\
	1 & X_n
\end{array}
\right]
\end{equation}


\noindent The $\beta$ vector is, 
\begin{equation}\beta_{2 \times 1} =
\left[
\begin{array}{c}
	\beta_0 \\ 
	\beta_1 
\end{array}
\right]
\end{equation}

\noindent And the $\epsilon$ vector is,
\begin{equation} \epsilon_{n \times 1} =
\left[
\begin{array}{c}
	\epsilon_0 \\ 
	\epsilon_1 \\
	\vdots \\
	\epsilon_n 
\end{array}
\right].
\end{equation}

\noindent This can be written in matrix terms with a dot product, 
\begin{equation} Y_{n \times 1}=X_{n \times 2} \cdot \beta_{2 \times 1} + \epsilon_{n \times 1} ,
\end{equation}

\noindent where, \newline
\textbf{$Y$} is a vector of response \newline
\textbf{$\beta$} is a vector of parameters \\
\textbf{$X$} is a matrix of constants, called det design matrix\\
\textbf{$\epsilon$} is a vector of independent normal random variables with expectation\\

\noindent This can be shown in columns,
\begin{equation}
\left[
\begin{array}{c}
	Y_1 \\ 
	Y_2 \\ 
	\vdots \\
	Y_n 
\end{array}
\right]
=
\left[
\begin{array}{cc}
	1 & X_1 \\ 
	1 & X_2 \\ 
	\vdots & \vdots \\
	1 & X_n
\end{array}
\right]
\left[
\begin{array}{c}
	\beta_0 \\ 
	\beta_1 
\end{array}
\right]
+
\left[
\begin{array}{c}
	\epsilon_0 \\ 
	\epsilon_1 \\
	\vdots \\
	\epsilon_n 
\end{array}
\right].
\end{equation}

\noindent If the dependent variable $Y$ has more than one independent variable in a linear model, the equation looks like this, 

\begin{equation} Y_i = \beta_0 + \beta_1 X_{i1} + \beta_2 X_{i2} + ... + \beta_p-1 X_{i, p-1} + \epsilon_i .
\end{equation}

\noindent In matrix terms it is,  
\begin{equation}
\left[
\begin{array}{c}
	Y_1 \\ 
	Y_2 \\ 
	\vdots \\
	Y_n 
\end{array}
\right]
=
\left[
\begin{array}{cccc}
	1 & X_{11} & ... & X_{1, p-1} \\ 
	1 & X_{21} & ... & X_{2, p-1} \\ 
	\vdots & \vdots &  & \vdots \\
	1 & X_{n1} & ... & X_{n, p-1}
\end{array}
\right]
\left[
\begin{array}{c}
	\beta_0 \\ 
	\beta_1 \\
	\vdots \\
	\beta_p-1 
\end{array}
\right]
+
\left[
\begin{array}{c}
	\epsilon_0 \\ 
	\epsilon_1 \\
	\vdots \\
	\epsilon_n 
\end{array}
\right].
\end{equation}

\noindent The $Y$ and $\epsilon$ vectors are the same as in the simple linear regression matrix. The $\beta$ vector has additional parameters, and the $X$ matrix now has a column of $n$ observations for each $p-1 X$ variables. 

\subsubsection{Polynomial Regression}
Polynomial regression models the relationship like this, 
\begin{equation}
	Y=\beta_0 + \beta_1 x + \beta_2 x^2	+ ... + \beta_{p-1} x^{p-1}+ \epsilon .
	\end{equation}

\noindent The coefficients can still be found through the method of least squares,

\begin{equation}
	Q=\sum_{i=1}^{n}(Y_i -(\beta_0 + \beta_1 X_i + \beta_2 x_i^2 + ... + \beta_{p-1}X_{i}^{p-1}))^2 .
\end{equation}
\newline

\noindent This can be written in matrix terms as,
\begin{equation}
	Q=(Y-X\beta)' (Y-X\beta) 
\end{equation}

\noindent Where the design matrix for polynomial regression is, 

\begin{equation}
	 X=
\left[
\begin{array}{ccccc}
	1&x_1&x_1^2&...&x_1^{p-1}\\ 
	1&x_2&x_2^2&...&x_2^{p-1} \\
	\vdots & \vdots &\vdots &&\vdots\\
	1&x_n&x_n^2&...&x_n^{p-1} 
\end{array}
\right].
\end{equation}

\noindent We can expand the expression from before, so it looks like this,
\begin{equation}
	Q=Y' Y -Y' X \beta -\beta' X' Y + \beta' X' X \beta .
\end{equation}


\noindent It is possible to find the value of $\beta$ that minimizes Q by differentiating with respect to $\beta_0$ and $beta_1$,

\begin{equation}
\frac{\partial}{\partial \beta}(Q)=
\left[
\begin{array}{c}
	\frac{\partial Q}{\partial \beta_0}\\ 
	\frac{\partial Q}{\partial \beta_1}
\end{array}
\right].
\end{equation}

\begin{equation}
\frac{\partial}{\partial \beta}(Q)=-2 X' Y + 2X' X \beta .
\end{equation}

\noindent Then minimum is found by calculating, where the gradient is 0.
\begin{equation}\b-2 X' Y+ 2X' X \beta =0,\end{equation}
\begin{equation} 2X' X \beta = 2X' Y,\end{equation}
\begin{equation}(X' X)^{-1} X' X \beta = (X' X)^{-1} X' Y,\end{equation}
\begin{equation} \beta=(X' X)^{-1} X' Y.\end{equation}
  For this to be calculated, the matrix $X' X$ has to be invertible, so the columns of $X$ have to be linearly independant.
 This means, that the assumption of no multicollinearity has to be met.
 
 \noindent So, 
 
\begin{equation}
 	X' Xa  =0
\end{equation}
\begin{equation}
 	a' X' X a =0
\end{equation}
\begin{equation}
 	(Xa)'(Xa)=0
\end{equation}
 
 \noindent This shows, that $X'Xa=0$ if and only if $(Xa)'(Xa)=0$, so $Xa=0$. 
 
 \noindent If it is supposed, that there exists a non-trivial solution for $Xa=0$, there also exists a non-trivial solution for $X'Xa=0$. It is therefore only invertible when the null space of $X$ is 0, or if the columns are linearly independent. This is called no multicollinearity, since there is independence between the independent variables. 
 
\noindent To ensure that the solution gives a minimum, the hessian matrix is calculated,
\begin{equation}
 \frac{\partial^2 Q}{\partial \beta' \beta}=2X'X
\end{equation} 
 \noindent If it is positive definite, the function Q has a global minimum, which can be shown with,
\begin{equation}
 a' (X' X)a = (Xa)' (Xa)
\end{equation}  
 \noindent As shown in $(63)$,
 these can only be calculated coefficients, if $Xa \not= 0$. This means, that $X'X$ is positive definite in all relevant situations, which is when the coefficients can be calculated. 
 
 \noindent The solution to the least squares method results in only containing $X'X$ and $X'Y$.
 The matrices are,
 
 
\begin{equation}
 X=
 \left[
 \begin{array}{ccc}
 	1 &X_1	& X_1^2\\ 
 	1 & X_2 & X_2^2	\\
 	\vdots & \vdots & \vdots \\
 	1 & X_n& X_n^2
 \end{array}
 \right]
\end{equation}
 
 \begin{equation}
 	X'X=
 \left[
 \begin{array}{ccccc}
 	n &\sum X_i	& \sum X_i^2 & ... &\sum X_i^{p-1}\\ 
 	\sum X_i & \sum X_i^2 & \sum X_i^3&...& \sum X_i^p	\\
 	\sum X_i^2 & \sum X_i^3 & \sum X_i^4 & ... & \sum X_i^{p+1}\\
 	\vdots&\vdots&\vdots& & \vdots	\\
 	\sum X_i^{p-1} &\sum X_i^p& \sum X_i^{p+1} & ... & \sum X_i^{2p-2}
 \end{array}
 \right]
\end{equation}
 
 \begin{equation}
 	X' Y=
 \left[
 \begin{array}{c}
 	\sum Y_i\\ 
 	\sum X_i Y_i	\\
 	\sum X_i^2 Y_i \\
 	\vdots	\\
 	\sum X_i^{p-1} Y_i
 \end{array}
 \right]
\end{equation}
 
 \noindent These are the necessary matrices to use the least squares method.

	\subsubsection{Assumptions}
	When making a regression its important to understand, that the regression has assumptions that needs to be fulfilled if the statistical conclusions are to be correct. If these assumptions are not upheld, it will create bias that will skew the results of the model.
\subsubsubsection{Homoscedasticity}
One of the assumptions of a polynomial regression is that homoscedasticity is fulfilled. Homoscedasticity is the assumption of constant error variance, where observations in a dataset would exhibit errors that have roughly the same spread across all levels of the independent variable.
\newline
If this assumption is not upheld, then this will cause the standard error to be biased and therefore not trustworthy. This problem causes further testing involving this standard error to become wrong, an example is the hypothesis test.
\newline
The reason for the assumption needs to be upheld, comes from how the regression is created. The regression is created via the ordinary least square method, that requires the assumption of homoscedasticity to be upheld.
\newline
A way to display homoscedasticity is through the variance-covariance matrix. The matrix shows whether the data contains homoscedasticity or heteroscedasticity through the diagonal values, that is the variance for the error in each independent variable in the regression. The variances in the matrix are calculated through an alternate equation from the one presented in \autoref{sec:disc}.
\begin{equation}
	\text{Var}(X) = \frac{1}{n - 1} \sum_{i=1}^{n} (x_i - \bar{x})^2
\end{equation}
The equation differs in that it excludes the weighting function and it assumes equal probability for each data point. These changes stems from the data deriving from a sample rather than the population. Consequently the use of $\frac{1}{n-1}$ rather than $\frac{1}{n}$, is to account for the bias correction. This correction is essential, as it prevents underestimation when working with samples rather than populations, by addressing bias.\\
If the matrix contains approximately the same values through the diagonal, then the assumption of homoscedasticity is upheld, else the data contains heteroscedasticity. This is a showcase of the variance-covariance matrix with homoscedasticity:


	
\begin{equation}
Var(\varepsilon) = \mathbb{E}[\boldsymbol{\epsilon} \boldsymbol{\epsilon}'] = 
\left[
\begin{array}{cccc}
	\sigma^2 & 0        & \cdots & 0 \\
	0        & \sigma^2 & \cdots & 0 \\
	\vdots   & \vdots   & \ddots & \vdots \\
	0        & 0        & \cdots & \sigma^2 \\
\end{array}
\right]
\end{equation}


\noindent The general way of writing the variance-covariance matrix, when the mean of the error i assumed to be $0$, is by this formula:
\begin{equation}
Var(\varepsilon) = \mathbb{E}[\boldsymbol{\epsilon} \boldsymbol{\epsilon}'] = 
\left[
\begin{array}{cccc}
	\sigma^2_{11} & \sigma_{21} & \cdots & \sigma_{i1} \\
	\sigma_{12} & \sigma^2_{22} & \cdots & \sigma_{i2} \\
	\vdots      & \vdots      & \ddots & \vdots      \\
	\sigma_{1j} & \sigma_{2j} & \cdots & \sigma^2_{ij}
\end{array}
\right]
\end{equation}

\noindent Every position in the matrix is calculated, then the diagonal will tell if the data contains homoscedasticity or heteroscedasticity. The off-diagonal values represent covariances between error terms. While not the focus of this project, non-zero values here may indicate correlation between errors, which would violate the assumption of independence \cite{Heteroscedasticity} . \newline 

\noindent \textbf{Detecting Heteroscedasticity} \\
When working with data, one method of checking for heteroscedasticity is to visualize it through plotting the residuals, but its not always possible to detect heteroscedasticity through visual media. Another approach is to calculate if the data contains heteroscedasticity. This can be done through the Breusch-Pagan test, that is specifically designed to detect heteroscedasticity. The test detects heteroscedasticity, by regressing the residuals on the independent variables and checks if the independent variables have an effect on the residual variance. If this is not the case, then there is no heteroscedasticity. This checked through a hypothesis test, where the $H_0$ is that the data contains homoscedasticity and $H_a$ is that the dataset contains heteroscedasticity \cite{HomoSce}.

\subsubsubsection{No Multicollinearity}
Perfect multicollinearity is a term used for describing a perfect linear relationship between two or more independent variables. This relationship occurs when an independent variable can be perfectly predicted from other independent variables. In mathematical terms, this could be written as a linear regression:
\begin{equation}
X_1 = \alpha+\beta_1\cdot X_2+...+\beta_n\cdot X_n
\end{equation}
Where $X_1...X_n$ is all the independent variables that have a perfect linear relationship. The coefficients are represented by $\beta_1...\beta_n$ and they are the amount that $X_1$ changes when their relative independent variable changes.
Lastly $\alpha$ is the intercept and represents the value of $X_1$, when all other independent variables are zero.
\newline
The regression model can feel the effects of multicollinearity even without there being perfect multicollinearity. A strong linear relationship is enough to have an effect on the model. The problem caused by multicollinearity, is that as it increases the variance of the value that the coefficients can receive also increases. Where as perfect multicollinearity will make the model unable to estimate a value of one coefficient, due to the perfect linearity between the independent variables \cite{MultAndMis}.
\newline

\noindent \textbf{Detecting Multicollinearity}\\
A common method to detect multicollinearity in a dataset, is to compute the pearson´s correlation coefficient for all combinations of independent variables. This will result in the correlation matrix, as seen in \autoref{fig:1}, where each cell represents the strength and direction of the relationship between two independent variables. The Pearsons correlation coefficient is calculated through this formula:
\begin{equation}
r = \frac{n \sum xy - (\sum x)(\sum y)}{\sqrt{[n \sum x^2 - (\sum x)^2][n \sum y^2 - (\sum y)^2]}}
\end{equation}
Where $n$ is the number of observations, with $x$ and $y$ representing the two variables tested for correlation and the pairwise correlation coefficient is denoted as $r$. When computing the value of $r$, the value will be in a range: $-1\leq r \leq 1$. If the value of $r$ is $-1$ or $1$, that indicates a perfect either negative or positive correlation and if the value is $0$, then there is no correlation between the variables. The correlation matrix has a connection to the variance-covariance matrix, by the correlation matrix being a normalized version of the variance-covariance matrix. This means that for each spot in the variance-covariance matrix, the value is divided by the product of the standard deviations of the corresponding independent variables. By normalizing the values, then the values will go from $-1$ to $1$.\cite{DetectMulti} \newline


\noindent\\\\
To test the consequences of violating the assumption of multicollinearity, a metric to determine model complexity is needed. This is because a more complex model, given the same parameters plus an additional one, will always be more accurate than the same model with fewer parameters. The question of model complexity is too large for the scope of this assignment. Therefore, the consequences of violating the assumption of multicollinearity will not be further investigated. Instead, we will focus on the consequences of violating the assumption of homoscedasticity. \newline


