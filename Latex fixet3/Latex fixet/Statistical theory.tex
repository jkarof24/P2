This section will focus on the theoretical background needed to understand how to create regression models, how to test a model's reliability and significance.

\noindent The theory in sections 3.1 through 3.5 is based on the book 'Probability and Statistics for Engineers and Scientists' \cite{ProbAndStat}. \newline

\noindent Statistics is a field in mathematics, that gives tools to analyse and understand data. Statistics comes in two branches, these are descriptive statistics and inferential statistics. Descriptive statistics is used to describe the general tendencies in some data, such as finding the mean and variance, but no predictions or conclusion are made beyond the data itself. In contrast, inferential statistic is the branch of statistics that makes predictions and generalizations of a population based on a sample, this includes methods such as hypothesis testing and regressions. 

\noindent To address the challenges, when violations in the assumptions of modeling regressions through classical means occur, a fundamental statistical understanding is needed. This section introduces the foundational concept of statistics that is required to understand regression models.


\subsection{Probability space}

The \textbf{sample space}, \textit{S}, is the set of all possible outcomes.
\newline
If a sample space contains a finite number of possibilities or an unending sequence with as many elements as there are whole numbers, it is called a \textbf{discrete sample space}.
\newline
Example: When rolling a standard six-sided die form the discrete sample space, the possible outcomes are $S={1,2,3,4,5,6}$ 
\newline

If a sample space contains an infinite number of possibilities equal to the number of points on a line segment, it is called a \textbf{continuous sample space}.
\break
Example: Measuring the heights of people in a population. This is a continuous sample space, because height can take any real value within a given range. 
\newline
An \textbf{event} is a subset, $A\subseteq S$, of the sample space. The event is the amount that contains all possible events.
An example of a discrete event could be rolling a die and getting an uneven number, this would be the event $A={1,3,5}$.
\newline 
For the continuous event, it could be that a person is between 160 cm and 170 cm tall.
\newline

The probability of an event \textit{A}, $P(A)$, is the sum of the weights of all sample points in \textit{A}.
The probability of the whole sample space is 1, $P(S)=1$
The probability of any event being between 0 and 1,$0<P(A)<1$
The probability of the empty set being 0, $P(Ø)=0$
\newline
\newline
Probability of mutually exclusive events
\newline
If \textit{A} and \textit{B} are mutually exclusive, $A \cap B=Ø$, then
\newline
$P(A \cup B) = P(A)+P(B)$
\newline
\newline
Where \textit{A} and \textit{B} never occur at the same time, so their union is equal to the two events added together. 
\newline
Probability of union
\newline
$P(A \cup B)=P(A)+P(B)-P(A \cap B)$
\newline
Here, the union of the two events is \textit{A} added to \textit{B}, but minus their common event, since it otherwise would be added twice. 
\newline
Two events \textit{A} and \textit{B} are independent, if 
\newline
$P(A|B)=P(A)$
\newline
The equivalent definition to this is:
\newline
Two events \textit{A} and \textit{B} are independent if and only if 
\newline
$P(A \cap B)=P(A)P(B)$
This says that the probability of both event \textit{A} and \textit{B} happening, is equal to the product of the two events.


\subsection{Random Variables}
Now we will introduce random variables, which are important to understand when analysing assumption violations. In statistics, data and results can vary. To understand and handle this randomness, we use random variables, to help describe this uncertainty. \newline

\noindent A random variable is defined as a function that associates a real number with each element in the sample space. We use capital letters to denote a random variabel, for example $X$, and then the corresponding small letter, in this case $x$, for one of its values. As an example we roll a dice 3 times, which gives us a sample space of the different combinations. Each point in the sample space gets a numerical value assigned between 0 and 3.For example, if the random variabel $X$ assumes the number of 5's rolled, then worst case is zero 5's rolled, and best case is three 5's rolled. These values are random quantities assumed by the random variabel $X$, and they are written like this: $X(5,1,2) = 1$ and $X(3,6,1) = 0$.
\newline

\noindent A random variable $X$ can be discrete, which means that its set of possible outcomes is countable. The dice example is a discrete random variable, because you can count how many times 5 is rolled. The outcomes of some statistical experiments may be neither finite nor countable. For example when something is measured such as temperature or speed where the set of possible values is an entire interval of numbers, it is not discrete. The random variable $X$ then takes values on a continues scale, which therefore is called a continuous random variable.

\subsubsection{Discrete Random Variable}
\label{sec:disc}
A discrete random variable can take each of its values with a certain probability. Frequently, it is convenient to represent all the probabilities of a random variable $X$ by a formula. Let $X$ be a discrete random variable which can take the values $x_{1}, x_{2},...$ Then the distribution of $X$ is given by the probability function:

\begin{equation}
	f(x_{i})=P(X=x_{i}),\quad i=1,2,...
\end{equation}
\newline
For a discrete random variable this function is also called the probability mass function, where following holds for each possible outcome $x$:

\begin{itemize}
	\item $P(X = x) = f(x).$
	\item $f(x) \geq 0,$
	\item $\sum_x f(x) = 1.$
\end{itemize}

\noindent In addition to the probability mass function $f$, the discrete random variable $X$ also has a cumulative distribution function $F(x)$ given by:

\begin{equation}
F(x) = P(X \leq x) = \sum_{x_i \leq x} f(x_i), \quad x \in \textbf{R}.
\end{equation}


\noindent This helps decide the probability that the random variable assumes a value equal to or smaller than $x$. Its sums up the probability density functions values.
\newline

\noindent The mean of a discrete variable $X$, with a distribution function $f(x_{i})$ is given by:

\begin{equation}
\mu = E(X) = \sum_i x_i P(X = x_i) = \sum_i x_i f(x_i).
\end{equation}

\noindent The mean is typically the expected value. It is a weighed average of the possible values of $X$. The values are weighed by its probability in the sample space.
\\

\noindent In addition to the mean, we should also mention the variance. The variance is the mean squared distance between the values of the variable and the mean value. It is given by:

\begin{equation}
\sigma^2 = E\left[(X - \mu)^2\right] = \sum_{i} (x_i - \mu)^2 f(x_i)
\end{equation}

\noindent The variance indicates whether the values of $X$ are far from the mean values or close. A high variance means that the values of $X$ have a high probability of being far from the mean values and vice versa. Along with the variance, the standard deviation is also often used. It is given by the square root of the variance:
\begin{equation}
\sigma=+\sqrt{\sigma^2}.
\end{equation}

\noindent The advantage of the standard deviation over the variance is that it is measured in the same units as $X$.

\subsubsection{Continuous Random Variable}
Contrary to a discrete random variable, a continuous random variable can take values that are not countable. A continuous random variable can take infinetly many possible values within a certain range or interval. For a continuous random variable $X$ the distribution is given by the probability density function $f$, which satisfies:

\begin{itemize}
	\item $f(x)$ is defined for all $x$ in $\textbf{R}$,
	\item $f(x) \geq 0$ for all $x$ in $\textbf{R}$,
	\item $\int_{-\infty}^{\infty} f(x) \, dx = 1.$
\end{itemize}

\noindent Condition 3. ensures that $P(-\infty < X < \infty) = 1$, which means that the probability of the random variable $X$ being between $-\infty$ and $\infty$ is 100\%. Furthermore the probability of $X$ assuming a specific value $a$ is zero, in other words: $P(X=a)=0$. That means that the values of the density function should not be interpreted as a probability of a given outcome. Instead the probability of $X$ is found by integrating over the probability density function. So, the probability that a continuous random variable $X$ lies between the values $a$ and $b$ is: 

\begin{equation}
P(a < X < b) = \int_a^b f(x) \, dx.
\end{equation}


\noindent A continuous random variable $X$ also has a distribution function $F(x)$, that also predicts whether $X$ assumes a value equal to or smaller than $x$. For a continuous random variable it is again given by integrating over the probability density function in the interval from $-\infty$ to $x$:
\begin{equation}
F(x) = P(X \leq x) = \int_{-\infty}^{x} f(y) \ dy.
\end{equation}


\noindent That also means $P(a<X<b)$ can be calculated by $F(b)-F(a)$.
\\

\noindent For a continuous random variabel $X$ the mean, variance and standard deviation the same interpretation applies. Just given by different formulars, which are:
\begin{equation}
	\mu = E(X) = \int_{-\infty}^{\infty} x f(x) \, dx
\end{equation}

\noindent and
\begin{equation}
\sigma^2 = E\left[(X - \mu)^2\right] = \int_{-\infty}^{\infty} (x - \mu)^2 f(x) \, dx,
\end{equation}

\noindent (The standard deviation is still given by til square root of the variance).


\subsection{Estimator and estimates}
If we are interested in certain parameters of a population distribution, we can look at a sample. From this, we can make a point estimate. 
\newline
Examples of this are, 
\newline
$\bar{x}$ is a point estimate of $\mu$
\newline
s is a point estimate of $\sigma$
\newline

\noindent This is often supplemented with a confidence interval.
\newline
This is an interval around the point estimate, where we are confident that the population parameter is located.
\newline

\noindent For $\mu$, we have different ways of estimating it. We can use the sample mean $\bar{X}$, or the average $X_T$ of the sample upper and lower quartiles. 
But in this case, we have to look out for bias. If the distribution of a population is skewed, then $X_T$ is biased. The result of this is, that in the long run, this estimator will systematically over or under estimate the value of $\mu$. This is written as,
\newline
$E(X_T) \neq \mu$.
\newline
It is generally preferred that the estimator is unbiased. In this case, $\bar{X}$ is an unbiased estimate of the population mean $\mu$.
\newline

\noindent The standard error of $\bar{X}$ is $\frac{\sigma}{\sqrt{n}}$. Here, the standard error decreases, when the sample size increases. If an estimator has this property, it is called consistent. If we compare, the estimator $X_T$ is also consistent, but has a greater variance than $\bar{X}$. 
\newline
It is generally preferred that the estimator has the smallest possible variance, and in that case it is called efficient. So $\bar{X}$ is an efficient estimator.
\newline
When estimating a parameter, the symbol $\hat{}$ is used above it. For $\mu$, $\hat{\mu} = \bar{X}$ .
\newline
We can calculate $\bar{X}$ using the following formula,
$$\bar{X}=\frac{1}{n} \sum_{i=1}^{n}X_i$$   
\newline
For the variance $\sigma$, we can estimate it by using the formula for $S^2$,
$$S^2=\frac{1}{n-1} \sum_{i=1}(X_i-\bar{X})$$
\subsection{Probability distribution}
Data can come in various distributions depending on different parameters such
as degrees of freedom. The distribution is the shape of the data and it will have
an effect on statistical models. Therefore it is important to have an understanding of distributions.
\subsubsection{Normal distribution}
In the world of statistics, the most common distribution is the normal distribution. It is constructed as a bell shape. The normal distribution is a continuous distribution, with this density function:
$$n(x;\mu,\sigma) =\frac{1}{\sqrt{2\pi\sigma}}e^{-\frac{(x-\mu)}{2\sigma^2}}$$
The distribution is dependent on the mean($\mu$) and the standard deviation($\sigma$), where changes to the mean will result in a change in the positioning of the normal distribution. Whereas a change in the standard deviation will change the spread of the curve. The normal distribution also always contains an area under the curve that is equal to one. This is to ensure that the normal distribution correctly models probability.

There is a special case of the normal distribution, called the standard normal distribution, where the mean is zero and the standard deviation is one. All variations of a normal distribution can be standardized by a transformation of the distribution, using the Z-score formula.
\newline
$$Z=\frac{X-\mu}{\sigma}$$
\newline
$Z$ in the Z-score represents the amount of standard deviations a given $X$ value, deviates from the mean.

\subsubsection{The central limit theorem}
A very effective theorem in statistics is the central limit theorem. This theorem states that if a random sample $\overline{X}$, with the size $n$, is taken from a population with a mean and a finite variance, then as $n$ goes towards infinity, the distribution will resemble a normal distribution. If used with the Z-score formula, the distribution will resemble a standard normal distribution. The formula for the Z-score, when in conjunction with the central limit theorem, looks like this:
$$Z=\frac{\overline{X}-\mu}{\sigma/\sqrt{n}}$$
Where $\overline{X}$ is a random sample of size $n$ and $\mu$ is the mean of the true population. The standard error i represented by $\sigma/\sqrt{n}$, where $\sigma$ is the standard deviation and $n$ is the sample size.
Usually the standard deviation is unknown, for these situations it's possible to use the estimator $S^2$. This estimates the variance of the population from the variance of the sample, by this formula:
$$s^2=\sum_{i=1}^{n}\frac{(x_{i}-\overline{x})}{n-1}$$

The square root of the variance is the standard deviation, therefore the square root of the estimator $S^2$ would be the estimated standard deviation. The problem with using the estimator $S^2$, is that with small samples the variance is small and therefore it contains a lot of bias. In this situation the t-distribution would be used instead of the normal distribution, because the t-distribution takes the bias into account the bias of the standard deviation. It does this by having thicker tails, meaning that the probability of more extreme values are higher.

\subsubsection{The t-distribution}
The t-distribution is shaped as the standard normal distribution, in a bell shape and symmetrical around the mean of zero, the difference is that the t-distribution is more variable. This comes from the fact that the t-distribution is dependent on the degrees of freedom. When the degrees of freedom surpasses 30, the rule of thumb is that the distribution will resemble a normal distribution. So before 30 degrees of freedom, the distribution contains more variance.
The t-distribution will come to resemble the standard normal distribution, when in surpasses 30 degrees of freedom, this makes sense, since the two distributions have the same formula:
$$T=\frac{\overline{X}-\mu}{S/\sqrt{n}}$$

The only difference is the estimated standard deviation $S$.
\newpage
\subsection{Statistical methods}

\subsubsection{Confidence intervals}
The confidence interval is a good tool to use, when trying to estimate a parameter of a population. Its used to create an interval, where the parameter has a probability to lie inside of. This probability is called the confidence level and it's a chosen value, usually the chosen confidence level is either 95\% or 99\%. The confidence interval will become bigger with a larger confidence level. A good confidence interval is small with a large confidence level, this will usually occur when the sample size is large. The chosen confidence level relates to an $\alpha$-value, where as an example the chosen confidence level is 95\%, then the $\alpha$-value would be 5\% or normally written as $0.05$. The $\alpha$-value will sometimes be needed to find the critical value, that is used to calculate the margin of error, as an example it's used when trying to find the critical value of the confidence interval, when working with a t-distribution.
\newline
To set up a confidence interval, the margin of error needs to be computed and then that will be both added and subtracted from the point estimate. This will give the values of the outer bounds of the interval. The margin of error is calculated from this formula:
$$Margin\_of\_error = critical\_value \pm standard\_error$$
\newline
The standard error will change depending on which parameter that the confidence interval is estimating, but the general formula for the standard error is:
$$\frac{\sigma}{\sqrt{n}}$$
\newline
An example of computing a confidence interval of the mean while working with a standard normal distribution, then the formula for the confidence interval would be this:

$$P(-z_{\alpha/2}<Z<z_{\alpha/2}) = 1-\alpha$$
\newline
Where $1-\alpha$ is the confidence level. As it's the mean that is being estimated, then instead of Z-score, then $\mu$ must be isolated and that is done by multiplying $\frac{\sigma}{\sqrt{n}}$ and subtracting $\bar{X}$ on all sides, then multiplying all side by $-1$ to remove the minus sign. So the formula for a confidence interval of the mean will look like this:

$$P(\bar{X}-z_{\alpha/2}\frac{\sigma}{\sqrt{n}}<\mu<\bar{X}+z_{\alpha/2}\frac{\sigma}{\sqrt{n}})=1-\alpha$$
\newline
This formula will give the upper and lower bounds of the confidence interval.\\

\noindent \textbf{The interpretation of a confidence interval}
\newline
To interpret a confidence interval, it would be incorrect to interpret the confidence level of some value $x$, as the probability of the true parameter being inside of the interval. The reason behind this is that the computed interval is static, so either the value $x$ is inside the interval or it's not. So the correct way of interpreting the confidence interval is by taking multiple samples and computing the confidence interval for all samples, then the value $x$ would reside inside 95\% of the confidence intervals.
\textbf{Kilde for fortolkningen af kofidense intervaller:}
\newline
$http://www.drhuang.com/science/mathematics/book/probability_and_statistics_for_engineering_and_the_sciences.pdf$


\subsubsection{Hypothesis testing}
A hypothesis test is used to test an assumption about a population. This is done from a sample of the population, as the information about the population is usually hard to come by. A hypothesis test is set up, by having a null hypothesis and an alternate hypothesis.
$$H_0 = Null\; hypothesis$$
$$H_a = Alternate\; hypothesis$$
When working with hypothesis testing, the hypothesis $H_0$ is usually represented as the status quo, where as the hypothesis $H_a$ is represented as the opposition. It is also important to note that there is only two outcomes of a hypothesis test, either $H_0$ is rejected in favor of $H_a$ or $H_0$ is failed to be rejected. Therefore in no situation can $H_0$ be stated to be an absolute truth, as there might be other samples where $H_0$ will be rejected. Therefore in a hypothesis test $H_0$ needs to be the thing that can be rejected and if $H_0$ gets rejected, then $H_a$ will become the new status quo until proven otherwise.\\
In a hypothesis test $H_0$ will be the assumption that a parameter for two populations is the same, where as $H_a$ can be either one of three assumptions, depending on the intention of the hypothesis test.
$$H_0: \theta = \theta_0$$
$$1.\;H_a: \theta \neq \theta_0$$
$$2.\;H_a: \theta < \theta_0$$
$$3.\;H_a: \theta > \theta_0$$
When the direction of the rejection is not important and also is unknown, then (1) will be the case. This scenario sets up a two-tailed-test, where the hypothesis test is used to reject $H_0$ if $H_a$ is either significantly larger or smaller than $H_0$, this means that the critical area is on both sides of the difference of $\theta$ and $\theta_0$. Either (2) or (3) will set up a one-tailed-test, where depending on what is important, either the hypothesis test is used to determine if $H_a$ is significantly bigger or smaller than $H_0$. This means that the critical area only spans one side of the difference between $\theta$ and $\theta_0$.\\

\newline
\noindent \textbf{Error in hypothesis testing}
\newline
When making a hypothesis test there is four different possible outcomes. The results are separated by correct decisions and errors. There exist two types of hypothesis errors, called type 1 error and type 2 error. The type 1 error occurs when $H_0$ is mistakenly rejected and $H_0$ is true. Type 2 error is the opposite, where $H_a$ is rejected and $H_a$ is true. The types of outcomes occurring from a hypothesis test can be seen in Table \ref{tab:example2x3}
\begin{table}[h!]
	\centering
	\begin{tabular}{|c|c|c|}
		\hline
		 & $H_0$ is true & $H_0$ is false \\
		\hline
		Does not reject $H_0$ & Correct decision & Type 2 error \\ \hline
		Reject $H_0$ & Type 1 error & Correct decision \\
		\hline
	\end{tabular}
	\caption{Outcomes of a hypothesis test}
	\label{tab:example2x3}
\end{table}

It is possible to compute the possibility of a type 1 error occurring, because the probability is equal to the significance level also denoted as $\alpha$. So when a significance level of 0.05 is chosen, then that is the same as the probability of a type 1 error occurring. As for computing the probability of a type 2 error occurring, it can only be done if $H_a$ is defined. The probability of a type 2 error occurring is denoted as $\beta$ and by a defined $H_a$, whats meant is that the $\mu$ of the sample is needed. Depending on the distribution and sample size, the calculation of $\beta$ will change.
As an example in a normal distributed sample, the $\mu$ is needed in calculating a Z-score and this Z-score is needed to extract a value from the Z-table. Then the probability of a type 2 error occurring is calculated from this formula:
$$
\beta = 1-\phi(Z_\beta)
$$
In this formula $\phi(Z_\beta)$ is the value from the Z-table. Its possible to reverse this formula and denote the probability of a correct decision, where $H_0$ is rejected and $H_a$ is true, this is denoted as, $1-\beta$.
It is possible to change the probability of these two errors occurring, this is done by changing the significance level. The consequence of this is that one of the errors will always lower its probability of occurring, while the other will have its probability increased. By reducing the significance level, the type 1 error will have a lower probability of happening, but a type 2 error will have an increase in probability of occurring. The opposite where the significance level is increased, the probability of the type 1 error will increase and for the type 2 error, it will be reduced. To overcome this problem, the only solution is to increase the sample size, as this will lower the probability of either of the two error types of happening.




