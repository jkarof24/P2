\subsection{Probability space}
To understand how testing models work, we first need to start at the basics. First off is probability space, and to understand probability space, we have to start with the sample space.

\noindent The sample space, \textit{S}, is the set of all possible outcomes, and it can be either discrete or continuous. If a sample space contains a finite number of possibilities or an unending sequence with as many elements as there are whole numbers, it is called a discrete sample space. For example, when rolling a standard six-sided die the discrete sample space is the possible outcomes: $S={1,2,3,4,5,6}$.
\newline

\noindent If a sample space contains an infinite number of possibilities equal to the number of points on a line segment, it is called a continuous sample space. For example when measuring the heights of people in a population. This is a continuous sample space, because height can take any real value within a given range.\newline

\noindent In the sample space we have events, $A\subseteq S$, which is a subset of the sample space. The event is the amount that contains all possible events. An example of a discrete event could be rolling a die and getting an uneven number, because the uneaven numbers are a subset of alle the possible outcomes from 1-6. This would be the event $A={1,3,5}$. 
\newline 
For the continuous event, it could be that a person is between 160 cm and 170 cm tall.
\newline

\noindent The probability of an event \textit{A}, $P(A)$, is the sum of the weights of all sample points in \textit{A}.
The probability of the whole sample space is 1, $P(S)=1$.
The probability of any event being between 0 and 1,$0<P(A)<1$.
The probability of the empty set being 0, $P(Ø)=0$.
\newline
\newline

\noindent If \textit{A} and \textit{B} are mutually exclusive, $A \cap B=Ø$, then
\newline
$P(A \cup B) = P(A)+P(B)$,
\newline
\newline
where \textit{A} and \textit{B} never occur at the same time, so their union is equal to the two events added together. 
\newline

\noindent We have the probability of union,
\newline
$P(A \cup B)=P(A)+P(B)-P(A \cap B)$,
\newline
here, the union of the two events is \textit{A} added to \textit{B}, but minus their common event, since it otherwise would be added twice. 
\newline
\noindent Two events \textit{A} and \textit{B} are independent, if 
\newline
$P(A|B)=P(A)$.
\newline
The equivalent definition to this is:
\newline
\noindent Two events \textit{A} and \textit{B} are independent if and only if
\newline
$P(A \cap B)=P(A)P(B)$.
This says that the probability of both event \textit{A} and \textit{B} happening, is equal to the product of the two events.
