Regression models are powerful tools for drawing statistical inferences from data. They serve as a gateway to more advanced statistical analysis and inference. However, despite their power, regression models are also quite fragile. This fragility stems from the assumptions that must be satisfied in order to avoid misleading or invalid conclusions. In this project we will focus on the effects of Monte Carlo Bootstrapping and how it can be used to overcome assumption violations, specifically violation in the assumption of homoscedasticity. We determine the effects of Monte Carlo Bootstrapping by resampling synthetic data, that contains multicollinearity. Two polynomial regression models are then created: one utilizing the resampled data and the other derived from the original data. The two models are compared in metrics such as root mean squared error and mean bias error. We conclude that the polynomial regression model derived from resampled data, has a greater accuracy. 