	\section{Pseudo Random Number Generator}
	
	To generate the dataset, we use a Pseudo-Random Number Generator (PRNG).
	PRNGs are algorithms that produce sequences of numbers that appear random
	but are actually deterministically generated from an initial seed value. While
	these numbers are not truly random, they are sufficiently unpredictable for many
	practical applications.
	Random numbers are widely used in fields such as statistics, game theory, cryp-
	tography, and simulations. These applications require numbers that behave
	as if they were random, yet can be reproduced when needed. This is where
	PRNGs come in—they allow for repeatable randomness, making them ideal for
	controlled experiments, testing, and security.
	This chapter will explore the key concepts behind PRNGs. Before going into the
	mechanics of these generators, it is important to first understand what ’random’
	means and the characteristics that define truly random numbers.
	
	\subsection{Properties of PRNGs}
	
The quality of a PRNG is determined by several key factors that influence its
use for different applications. Some of the properties of a good PRNG is prop-
erties: Independency, a large period and reproducibility
The numbers produced by the PRNG should be statistically independent, en-
suring that each generated value exhibits no correlation with previous numbers
or other sequences. This implies that knowledge of previously generated num-
bers or sequences provides no advantage in predicting the next output.
A PRNG operates within a specific interval before its sequence begins to repeat.
A high-quality PRNG has a long interval, delaying repetition and enhancing its
unpredictability. Conversely, a PRNG with a shorter period becomes more pre-
dictable and less suitable for practical use.
A key feature of a PRNG is its ability to reproduce the same sequence of num-
bers when given a specific seed. This property is particularly useful in testing
and simulation scenarios, where it is essential to generate identical sequences
multiple times for consistency and reproducibility
In addition, a PRNG must be fast and efficient to prevent it from introducing
performance bottlenecks within an application. The speed of number generation
directly impacts computational efficiency, especially in applications requiring a
large volume of random numbers. An inefficient PRNG can significantly slow
down processes, undermining the overall performance of the system. Therefore,
balancing randomness and efficiency is essential for practical applications
	
	\subsection{Linear Congruential Generator}
	
	Linear Congruential Sequence (LCS) is a commonly used approach to generate
	pseudo-random numbers. LCS generates a sequence of numbers using a linear
	recurrence relation LCS is expressed as:

		$$X_{n+1} = (aX_n + c) \bmod m.$$

	where $X_0$ is the seed, $a$ is the multiplier, $c$ is the increment, and $m$ is the modulus.
	
	\textbf{Example:} Given $a = 5$, $c = 1$, $m = 16$, and $X_0 = 7$:
		$$X_1 = (5 \cdot 7 + 1)\bmod 16 = 4 $$
		$$X_2 = (5 \cdot 4 + 1) \bmod 16 = 5 $$
		$$X_3 = (5 \cdot 5 + 1) \bmod 16 = 10 $$
		$$X_4 = (5 \cdot 10 + 1) \bmod 16 = 3 $$

	
	This sequence has a period of 16. In an LCG, the period can be as large as $m$, but choosing parameters carefully is crucial to achieving long periods. Therefore
