In statistics, when making a regression model through classical means, the model requires certain assumptions to be met. These assumptions are necessary for the model to produce correct statistical conclusions. Methods have been made to overcome this obstacle, one method in particular being bootstrapping. This method will be used in this project to show its effects on regression creation, when specifically the assumptions of homoscedasticity and no multicollinearity are broken. This project will therefore shed light on when resampling can be used to overcome assumptions of regressions and when it will fail.

\begin{itemize}
	\item How will breaking the assumptions, homoscedasticity and no multicollinearity in a classical polynomial regression, effect the performance of the model and how can this be solved by using Monte Carlo Bootstrapping?
\end{itemize}