Assumptions need to be met before proceeding with some calculations. Here are assumptions
\newline 

The assumption independence of errors states that the errors from the model are not correlated with each other but are independent. This means that one error cannot be used to predict the next one. \newline


 The assumption of linearity states that the relationship between the independent variables and the parameters, also known as the coefficients, is linear. This does not mean that the regression model itself has to be linear. For example, in polynomial regression, the relationship between the independent variables and the coefficients is still linear, but the independent variables can be transformed using powers.\newline


Homoscedasticity is the assumption of constant variance across errors for all levels of the independent variables. \newline

The assumption normality of errors states that the errors between the model and the observed values, also called residuals, are normally distributed. If this is not met, it may result in a biased model and a worse model fit. \newline


Multicollinearity occurs when two or more independent variables in a regression model are highly correlated with each other. This means that changes in one independent variable are associated with changes in another, making it difficult to determine the individual effect of each independent variable on the dependent variable. \newline

Correct model specifications assumes that the provided dependent variables for the models are the correct ones. If one is missing or the model is overfitting, this may result in incorrect coefficients and introduce errors into the model.

