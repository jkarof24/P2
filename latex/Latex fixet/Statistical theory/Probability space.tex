\subsection{Probability space}

The \textbf{sample space}, \textit{S}, is the set of all possible outcomes.
\newline
If a sample space contains a finite number of possibilities or an unending sequence with as many elements as there are whole numbers, it is called a \textbf{discrete sample space}.
\newline
Example: When rolling a standard six-sided die form the discrete sample space, the possible outcomes are $S={1,2,3,4,5,6}$ 
\newline

If a sample space contains an infinite number of possibilities equal to the number of points on a line segment, it is called a \textbf{continuous sample space}.
\break
Example: Measuring the heights of people in a population. This is a continuous sample space, because height can take any real value within a given range. 
\newline
An \textbf{event} is a subset, $A\subseteq S$, of the sample space. The event is the amount that contains all possible events.
An example of a discrete event could be rolling a die and getting an uneven number, this would be the event $A={1,3,5}$.
\newline 
For the continuous event, it could be that a person is between 160 cm and 170 cm tall.
\newline

The probability of an event \textit{A}, $P(A)$, is the sum of the weights of all sample points in \textit{A}.
The probability of the whole sample space is 1, $P(S)=1$
The probability of any event being between 0 and 1,$0<P(A)<1$
The probability of the empty set being 0, $P(Ø)=0$
\newline
\newline
Probability of mutually exclusive events
\newline
If \textit{A} and \textit{B} are mutually exclusive, $A \cap B=Ø$, then
\newline
$P(A \cup B) = P(A)+P(B)$
\newline
\newline
Where \textit{A} and \textit{B} never occur at the same time, so their union is equal to the two events added together. 
\newline
Probability of union
\newline
$P(A \cup B)=P(A)+P(B)-P(A \cap B)$
\newline
Here, the union of the two events is \textit{A} added to \textit{B}, but minus their common event, since it otherwise would be added twice. 
\newline
Two events \textit{A} and \textit{B} are independent, if 
\newline
$P(A|B)=P(A)$
\newline
The equivalent definition to this is:
\newline
Two events \textit{A} and \textit{B} are independent if and only if 
\newline
$P(A \cap B)=P(A)P(B)$
This says that the probability of both event \textit{A} and \textit{B} happening, is equal to the product of the two events.
