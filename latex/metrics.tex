\begin{document}
\section{Metrics}

When producing a synthetic dataset, there is a need for a lot of random numbers. Many programming languages have a built-in function that produces numbers that appear random, but actually are not. Computers are deterministic machines, and can therefore not produce a number without some sort of algorithm. With knowledge of this algorithm, it would be possible to predict the next number; hence the numbers are not completely random. Random numbers should be independent, i.e., the next number should not have any connection to the number or numbers produced previously. The distribution should also be uniform, meaning if you generate 1,000,000 random numbers in the range [0,1), you'd expect about 500,000 values in [0, 0.5) and about 500,000 in [0.5, 1). Earlier, these numbers have been produced by flipping coins or rolling dice. Now, it is possible to produce truly random numbers by using atmospheric noise. Despite its potential advantages, this method requires significant resources, making it inefficient for the intended application. Therefore, pseudo-random numbers will be used instead.

Pseudo-random numbers look and act like random numbers, but are actually deterministically generated from an initial seed value. While
these numbers are not truly random, they are sufficiently unpredictable for many practical applications.
To generate the data, we use a Pseudo-Random Number Generator (PRNG).
PRNGs are algorithms that produce sequences of numbers that appear random.


Random numbers are widely used in fields such as statistics, game theory, cryptography, and simulations. These applications require numbers that behave
as if they were random, yet can be reproduced when needed. This is where
PRNGs come in. They allow for repeatable randomness, making them ideal for
controlled experiments, testing, and security.
\\
This chapter will explore the key concepts behind PRNGs. Before going into the
mechanics of these generators, it is important to first understand what ’random’
means and the characteristics that define truly random numbers.

\subsection{R-Squared}

The quality of a PRNG is determined by several key factors that influence its
use for different applications. Some of the properties of a good PRNG is prop-
erties: Independency, a large period and reproducibility
\newline \\
The numbers produced by the PRNG should be statistically independent, en-
suring that each generated value exhibits no correlation with previous numbers
or other sequences. This implies that knowledge of previously generated num-
bers or sequences provides no advantage in predicting the next output.

\end{document}